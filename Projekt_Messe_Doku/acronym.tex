\begin{acronym}
	\acro{PT}[PT]{Packet Tracer}
	\acro{IP}[IP]{Internet Protokoll}: Ein verbindungsloses Protokoll, dass die Adressierung von Computern und das Routing von Paketen ermöglicht.
	\acro{wr}[wr]{Write memory}: Ein wichtiger Befehl, bei der Konfiguration von Routern im Packet Tracer, genutzt damit Konfigurationen persistent sind.
	\acro{RADIUS}[RADIUS]{Remote Authentication Dial-In User Service}: Client-Server-Protokoll zur Authentifizierung, Autorisierung und Accounting von Benutzern bei Einwahlverbindungen in Netzwerke.
	\acro{AP}[AP]{Access Point}: Schnittstelle für kabellose Kommunikationsgeräte.
	\acro{LWAP}[LWAP]{Light Weight Access Point}: Schnittstelle für kabellose Kommunikationsgeräte.
	\acro{CIDR}[CIDR]{Classless Inter-Domain Routing}: Verfahren zur effizienten Nutzung von IP-Adress-Räumen unter Verwendung von Subnetzmasken.
	\acro{DHCP}[DHCP]{Dynamic Host Configuration Protocol}: Ein Verfahren um von einem Server aus Netzwerkkonfigurationen an Clients zuzuweisen.
	\acro{WPA2}[WPA2]{Wi-Fi Protected Access 2}: Sicherheitsstandard für Funknetze basierend auf AES.
	\acro{AES}[AES]{Advanced Encryption Standard}: Blockchiffre zur Verschlüsselung von Daten.
	\acro{ACL}[ACL]{Access Liste}: Liste von zugelassenen und blockierten IP Adressen(-Bereichen).
	\acro{TCP}[TCP]{Transmission Control Protocol}: Ein zuverlässiges, verbindungsorientiertes, paketvermitteltes Layer 4 Netzwerk Protokoll.
	\acro{HTTPS}[HTTPS]{Hypertext Transfer Protocl Secure}: Ein Kommunikationsprotokoll im World Wide Web, das eine Transportverschlüsselung darstellt.
	\acro{JSON}[JSON]{JavsScript Object Notation}:
	Ein kompaktes Datenformat in einer einfach lesbaren Textform unabhängig von Programmiersprachen.
	\acro{SQL}[SQL]{Structured Query Language}: Eine Abfragesprache um auf Daten in einer relationalen Datenbank zuzugreifen und diese zu verwalten.
	\acro{UI}[UI]{User Interface}: Englische Bezeichnung für Benutzeroberfläche.
\end{acronym}